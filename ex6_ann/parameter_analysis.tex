\subsubsection{参数分析}

本实验对KDTree混合优化和LSH SIMD并行算法进行了参数敏感性分析,通过网格搜索揭示了关键参数对性能的影响。

\paragraph{KDTree混合优化参数分析}

主要参数:树数量(5-12棵)和搜索节点数(3000-12000)。实验结果如下:

\begin{table}[h]
\centering
\caption{KDTree混合优化参数配置性能对比}
\begin{tabular}{|c|c|c|c|c|c|}
\hline
\textbf{树数量} & \textbf{搜索节点数} & \textbf{召回率} & \textbf{查询延迟(μs)} & \textbf{加速比} & \textbf{性能得分} \\
\hline
5 & 3000 & 99.9\% & 1441.66 & 2.08x & 2.08 \\
8 & 5000 & 100.0\% & 2423.06 & 1.24x & 1.24 \\
10 & 8000 & 100.0\% & 3824.51 & 0.78x & 0.78 \\
12 & 12000 & 100.0\% & 6246.45 & 0.48x & 0.48 \\
\hline
\end{tabular}
\end{table}

\textbf{关键发现:}所有配置都能达到99.9\%-100\%召回率,5棵树+3000搜索节点为最佳平衡点(99.9\%召回, 2.08x加速)。

\paragraph{LSH SIMD并行参数分析}

主要参数:哈希表数量(60-150)、哈希位数(14位)、搜索半径(8-12)、候选点数(1500-2500)。实验结果如下:

\begin{table}[h]
\centering
\caption{LSH SIMD并行参数配置性能对比}
\begin{tabular}{|c|c|c|c|c|c|c|}
\hline
\textbf{表数量} & \textbf{哈希位数} & \textbf{搜索半径} & \textbf{候选点数} & \textbf{召回率} & \textbf{查询延迟(μs)} & \textbf{加速比} \\
\hline
60 & 14 & 8 & 1500 & 73.4\% & 1165.10 & 2.57x \\
90 & 14 & 10 & 2000 & 78.9\% & 1325.89 & 2.26x \\
120 & 14 & 10 & 2000 & 88.8\% & 2126.67 & 1.41x \\
150 & 14 & 12 & 2500 & 89.4\% & 2190.91 & 1.37x \\
\hline
\end{tabular}
\end{table}

\textbf{关键发现:}表数量对性能影响最显著,从60表到150表召回率提升16\%,但加速比从2.57x降至1.37x。

\paragraph{算法选择指导}

\begin{itemize}
    \item \textbf{高召回率}: KDTree混合优化(5-8棵树),可达99.9\%+召回率
    \item \textbf{高速度}: LSH SIMD并行(60-90表),可获得2.5x+加速比
    \item \textbf{平衡应用}: KDTree 5棵树配置或LSH 90表配置
\end{itemize}

通过参数分析验证了算法设计的有效性,为实际应用提供了明确的参数选择指导。 